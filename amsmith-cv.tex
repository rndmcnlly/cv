% amsmith-cv.tex 
% based on format at: http://mboedick.org/code/latex_resume_tips.php

\documentclass[10pt]{article}
\usepackage{fullpage}
\textheight=9.0in
\pagestyle{empty}
\raggedbottom
\raggedright
\setlength{\tabcolsep}{0in}
\begin{document}

\begin{tabular*}{6.5in}{l@{\extracolsep{\fill}}r}
\large \textbf{Adam Marshall Smith}  & (831) 295-2624 \\
5105 Cordoy Lane & adam@adamsmith.as \\
San Jose, CA 95124 &  \\
\end{tabular*}
\\

\vspace{0.5in}
{\large \textbf{Education}}

\begin{itemize}

    \item 
    \begin{tabular*}{6in}{l@{\extracolsep{\fill}}r}
        \textbf{University of California Santa Cruz} & Santa Cruz, CA \\
        Ph.D., Computer Science, in progress (fourth year) & \\
    \end{tabular*}
%  \begin{itemize}
%    \item \textbf{Notable courses}: advanced artificial intelligence, advanced machine learning, combinatorial algorithms, interactive narrative, bayesian statistics, computational photography, computer vision for mobile devices, game design seminar, programming languages, analysis of algorithms
%    \item \textbf{Notable course projects}: theoretical basis and method for computing transient global illumination, interactive fiction remake of existing artificial reality game, new (but unremarkable) algorithms for 3SAT, enhancing low-light photography using infrared illumination, data mining for Netflix Prize using additional IMDB metadata, reinforcement learning agent for a real-time video game, soft-body physics simulator with rich scripting language
%
%     \end{itemize} % ucsc grad details

    \item 
    \begin{tabular*}{6in}{l@{\extracolsep{\fill}}r}
        \textbf{University of California Santa Cruz} & Santa Cruz, CA \\
        B.S. with Honors, Computer Science (GPA 3.62), 2005 & \\
  \end{tabular*}
  
%  \begin{itemize}
%         \item Honors in the major and Comprehensive honors
%         \item \textbf{Notable courses}: computer architecture, advanced programming, computational models, analysis of algorithms, technical writing, compiler design, operating systems, artificial intelligence, game ai, ai for music, embedded software, comparative programming languages, advanced linear algebra, graphics, animation, scientific visualization, computer game design
%    \item \textbf{Notable course projects}: ad-hoc wireless router using LEGO Mindstorms, unified virtual memory and disk cache in DLXOS, rhythmic improvisation using binary pitch trees in LISP, peer-to-peer board game using sockets in Java, adaptive game playing agents in LISP, 3D flocking simulation in Python with OpenGL  
%     \end{itemize} % ucsc ugrad details
    
  \item 
    \begin{tabular*}{6in}{l@{\extracolsep{\fill}}r}
        \textbf{De Anza College} & Cupertino, CA \\
        Transfered Credits, 2003 & \\
  \end{tabular*}
%  \begin{itemize}
%         \item \textbf{Notable courses}: advanced C programming, advanced x86 assembly programming, all upper-level mathematics, all upper-level physics, experimental economics, spanish
%     \end{itemize} % deanza details

\end{itemize} % education

{\large \textbf{Research}}
\begin{itemize}

\item
    \textbf{Dissertation Research}
    \begin{itemize}
        \item Area: Artificial Intelligence
        \item Keywords: machine creativity, discovery systems, game design
        \item Statement: My research explores the intersection of machine creativity (which traditionally focuses on aesthetic artifact generation), discovery systems (which traditionally automate the scientific or mathematical discovery processes), and game design (which is traditionally carried out by particularly creative humans), with the aim of producing a systems that creatively discovers game design knowledge through experience (popping out sweet games along the way). Provocative, yeah?
    \end{itemize}
 
\item 
    \textbf{Publications}
    \begin{itemize}
        \item Pousman, Z., Romero, M., Smith, A., and Mateas, M. (September 2008) ``Living with Tableau Machine: A Longitudinal Investigation of a Curious Domestic Ingelligence'' In Proc. of the 10th Intl. Conference on Ubiquitous Computing (UbiComp 08).
        \item Smith, A., and Warmuth, M. K. (July 2008) ``Open Problem: Learning Rotations.'' In Proc. of the 12th Annual Conference on Learning Theory (COLT 08).
        \item Smith, A., Romero, M., Pousman, Z., and Mateas, M. (March 2008) ``Tableau Machine: A Creative Alien Presence.'' AAAI Spring Symposium on Creative Intelligent Systems. 
        \item Smith, A., Skorupski, J., and Davis, J. (February 2008) ``Transient Rendering'' Technical Report UCSC-SOE-08-26, School of Engineering, Univeristy of California, Santa Cruz.
  \end{itemize}
  
\item  
  \textbf{Selected Project Reports}
    \begin{itemize}
       \item Smith, A., and Skorupski, J. (2007) ``Transient Rendering.''
       \item Smith, A. (2007) ``Learning Transformations Between Directed Subspaces Online.''
       \item Smith, A. (2007) ``Selected Classical Problems from the History of Mathematics.''
       \item Smith, A., and Gunawardane, P. (2006) ``Genre-space Clustering of Users for the Netflix Prize.''
       \item Smith, A. (2006) ``Experience in the Game Design Seminar.''
       \item Scher, S., and Smith, A. (2005)``Night into Day: Enhancing Low-Light Color Photography.''
  \end{itemize}
    
\end{itemize} % Research

{\large \textbf{Teaching}}
\begin{itemize}

\item
    \begin{tabular*}{6in}{l@{\extracolsep{\fill}}r}
        \textbf{Teaching Assistantships} & UC Santa Cruz    \end{tabular*}
    
    \begin{itemize}
        \item Courses (deduped): ``Introduction to Computer Graphics'', ``Scientific Visualization and Computer Animation'', ``Game Engine Architecture'' and ``Computer Literacy''
        \item Earned \textit{2006 Outstanding Teaching Assistant Award} and \textit{2007 Excellence in Teaching Award (for Teaching Assistants)}
        \item Counseled students on large-scale, free-form, group projects
        \item Taught javascript programming to computer literacy students (without authorization!)
        \item Emphasized self-teaching, playful experimentation, and the use of awesome FOSS tools
        \item Continue to provide life-long graphics and game programming advice to past students
    \end{itemize}

\item
    \begin{tabular*}{6in}{l@{\extracolsep{\fill}}r}
        \textbf{Development} & UC Santa Cruz    \end{tabular*}
    
    \begin{itemize}
        \item Courses: ``Introduction to Computer Graphics'', ``Scientific Visualization and Computer Animation'', and ``Game Engine Architecture''
        \item Designed lectures, exams, homework, programming projects, and demos
        \item Developed new units for video game programing, shader programming, and raytracing 
        \item Created cross-platform, transparent, video game software template (used by 100+ students and multiple researchers to bootstrap game projects)
        \item Guided larger curriculum development discussions, maintaining focus on student engagement and long-term benefit
        \item Coached undergraduate volunteers
    \end{itemize}

\item 
  \begin{tabular*}{6in}{l@{\extracolsep{\fill}}r}
        \textbf{Guest Lectures} & UC Santa Cruz
    \end{tabular*}
    
    \begin{itemize}
    \item Digital Image Compositing
    \item Non-photorealistic Rendering
    \item Elementary Game Design
    \item Game Engine Architecture Spectrum
    \item Game Programming in Python for Non-Python Programmers
    \item Programmer-oriented Tools for Creativity in Graphics
    \end{itemize}
    
\item
  \begin{tabular*}{6in}{l@{\extracolsep{\fill}}r}
        \textbf{Education Talks} & Various Un-conferences, Sillicon Valley, CA
    \end{tabular*}
    
    \begin{itemize}
       \item New Foundations / Bayesian Reasoning and Geometric Algebra (EduCamp Stanford)
       \item Formal Language Skills for Reading, Writing, and Arithmetic (EduCamp Stanford)
       \item An Ecosystem of BarCamp-like Events (BarCamp Block)
  \end{itemize}
    
\end{itemize} % Teaching


{\large \textbf{Internships}}

\begin{itemize}

\item
  \begin{tabular*}{6in}{l@{\extracolsep{\fill}}r}
    \textbf{Software Engineering Intern} & Summer 2008\\
    Google, Inc. - Developer Tools  & Kirkland, WA\\
  \end{tabular*}
  
  \begin{itemize}
    \item Developed parser, name resolver, and type checker for Protocol Buffers
language
    \item Helped explore solutions for many-language parsing
    \item Contributed to design of \emph{large}-scale, distributed notgunnasay
  \end{itemize}

\item
  \begin{tabular*}{6in}{l@{\extracolsep{\fill}}r}
    \textbf{Software Engineering Intern} & Summer 2007\\
    Google, Inc. - Enterprise Engineering & Mountain View, CA\\
  \end{tabular*}
  
  \begin{itemize}
    \item Developed modules to expose GData services to Google Search Appliance
    \item Collaborated with technical writer on user-facing documentation
    \item Contributed to architecture debates for next-gen appliance
    \item Released modules as open source projects on Google Code
    \item Attended or watched 100+ Tech Talks on a variety of topics
  \end{itemize}

\item
  \begin{tabular*}{6in}{l@{\extracolsep{\fill}}r}
  \textbf{Staff Research Assistant} & Summer 2006 \\
  Los Alamos National Laboratory - High Performance Computing & Los Alamos, NM \\
  \end{tabular*}
  
  \begin{itemize}
    \item Integrated hardware-based, distributed, image compositing system into visualiztion software (paraview)
    \item Developed instruction-level-optimized software solution for comparisoni on cluster  (mpi/openib)
    \item Created interactive visualizations of huge materials science datasets (paraview)
    \item Organized experiments across a non-uniform cluster of eight nodes
    \item Performed technical demonstration for dignitaries
  \end{itemize}



\item
    \begin{tabular*}{6in}{l@{\extracolsep{\fill}}r}
        \textbf{Educational Associate} & Summer 2005 \\
        NASA Ames Research Center, Intelligent Robotics Group & Moffet Field, CA \\
    \end{tabular*}

    \begin{itemize}
    \item Developed and documented test procedures for PhaseSpace motion tracking system (\LaTeX)
    \item Evaluated system accuracy and precision (data collection, analysis)
    \item Created automated data analysis tool chain for sensor data (perl, gmake, gnuplot)
    \item Integrated sensor into test rover sensor system (c++, ice, java)
    \item Proposed new filtering method to improve sensor robustness (octave)
    \end{itemize}


\item
    \begin{tabular*}{6in}{l@{\extracolsep{\fill}}r}
        \textbf{Programmer / Other} & Summer 2004 \\
        Terracom Communications & Kigali, Rwanda \\
    \end{tabular*}  

    \begin{itemize}
    \item Created and scheduled several automated web scrapers and log analyzers (perl)
    \item Developed database driven mini-sites (php, templating libraries)
    \item Created full text search tool with stemming for multiple mini-sites (mysql)
    \item Researched sources and setup automated filtering for a dynamic tech-news site (perl)
    \item Setup and secured servers (linux)
    \item Proposed and implemented new web-caching policy yielding vastly increased performance, noticed by clients (squid)
    \item Worked 80+ hours/week in Kigali, meeting deadlines and completing side-projects
    \end{itemize}

\end{itemize} % internships

\newpage
{\large \textbf{High-Density Additional Information}}
\begin{description}
\item [Well-developed personal interests:] {\small computational light science (the intersection of computer graphics, computer vision, and physics), experimential electronic music, procedural visual media, next-next-gen web, network hacking, computer game design, geometry, abstract algebra, advanced physics, backpacking, snowboarding, mountain biking, philosophy of mathematics/science/systems/art/hacking, unconferences (*Camps), hackathons (SuperHappyDevHouse)}
\item [Some recreational projects (completed alone or with friends):] {\small invisible hand (prototype for document camera system that removes foreground objects from scene to improve readability), the.cubing.game (simple video game based on realistic physical simulation with original art, sound effects, and music), codepoem (a series of abstract, concrete code poems juxtaposing code for a context-free design grammar with one of its visual compositions), slut-o-meter (SafeSearch reporting gimmick), AjaxWar (distributed, multi-player,  real-time strategy game made using only javascript on web client before AJAXy support libraries existed), data visualizer for There.com participants, DriveByCTF (real-life wardriving game with centralized scoring through the web), spammer.pl (automation of laptop duties in wardriving game), GoGetter (javascript-based web crawler and collage generator), GifGif (a steganographic image codec), RadAudio (wavelet based audio codec with psychoacoustic modeling), markov.lisp (instant messaging log re-synthesizer), karamari.lisp (prototype for text adventure remake of the popular Katamari Damacy), POSIX Monkey Puncher (thread programming tutorial that is also silly game), Winter (music visualizer created at informal demo-party), cicada-client (WiFi reporting tool using DNS for clandestine upstream data transport)}
\item [Technical skill dump:] {\small html/xhtml/xml/xslt/css, sdl, opengl, osg, design patterns, prolog, matlab/octave, lisp/scheme, java/c++/c\#, python, ocaml, perl, php, make, bash, flex/bison, ms office, \LaTeX (for papers and presentations)}
\end{description}

{\large \textbf{References}}\\

\textit{Contact info available on request for:}\\
\begin{itemize}
  \item Michael Mateas (Research Advisor from UCSC)
  \item James Davis (Teaching Mentor from UCSC)
  \item Eric Haugh (Mentor from Google)
  \item Carolyn Connor-Davenport (Mentor from LANL)
  \item Terry Fong (Mentor from NASA)
  \item Jo\"{e}l Franusic (Friend, Coworker from Terracom, education/hacker/art philosophy discussion cohort)
  \item Jeff Lindsay (Friend, Founder of DevjaVu (a startup for which I am on the board of directors), education/hacker/systems philosophy discussion cohort)
\end{itemize}

{\large \textbf{Citizenship}}
\begin{itemize}
  \item United States (by birth)
\end{itemize}


\vspace{0.25in}
\center{\small Compiled \today.}

\end{document}

% vim: ts=4; tw=0; wrap
